\documentclass[twoside]{article}

\usepackage[sc]{mathpazo} % Use the Palatino font
\usepackage[T1]{fontenc} % Use 8-bit encoding that has 256 glyphs
\linespread{1.05} % Line spacing - Palatino needs more space between lines
\usepackage{microtype} % Slightly tweak font spacing for aesthetics

\usepackage[hmarginratio=1:1,top=32mm,columnsep=20pt]{geometry} % Document margins
\usepackage{multicol} % Used for the two-column layout of the document
\usepackage[hang, small,labelfont=bf,up,textfont=it,up]{caption} % Custom captions under/above floats in tables or figures
\usepackage{booktabs} % Horizontal rules in tables
\usepackage{float} % Required for tables and figures in the multi-column environment - they need to be placed in specific locations with the [H] (e.g. \begin{table}[H])
\usepackage{hyperref} % For hyperlinks in the PDF

\usepackage{lettrine} % The lettrine is the first enlarged letter at the beginning of the text
\usepackage{paralist} % Used for the compactitem environment which makes bullet points with less space between them

\usepackage{abstract} % Allows abstract customization
\renewcommand{\abstractnamefont}{\normalfont\bfseries} % Set the "Abstract" text to bold
\renewcommand{\abstracttextfont}{\normalfont\small\itshape} % Set the abstract itself to small italic text

\usepackage{titlesec} % Allows customization of titles
\renewcommand\thesection{\Roman{section}} % Roman numerals for the sections
\renewcommand\thesubsection{\Roman{subsection}} % Roman numerals for subsections
\titleformat{\section}[block]{\large\scshape\centering}{\thesection.}{1em}{} % Change the look of the section titles
\titleformat{\subsection}[block]{\large}{\thesubsection.}{1em}{} % Change the look of the section titles

\usepackage{fancyhdr} % Headers and footers
\pagestyle{fancy} % All pages have headers and footers
\fancyhead{} % Blank out the default header
\fancyfoot{} % Blank out the default footer
\fancyhead[C]{Bitquant Research Laboratories Working Paper No. 1} % Custom header text
\fancyfoot[RO,LE]{\thepage} % Custom footer text

\title{\vspace{-15mm}\fontsize{24pt}{10pt}\selectfont\textbf{A simple macroeconomic model of bitcoin}} % Article title

\author{
\large
\textsc{Joseph C Wang}
\normalsize Bitquant Research Laboratories\\
\normalsize \href{http://www.bitquant.com.hk/}{http://www.bitquant.com.hk/} \\
\normalsize \href{mailto:joequant@gmail.com}{joequant@gmail.com} \\
\date{2014 February 11}
}


\usepackage{natbib}
\bibliographystyle{plainnat}
\usepackage{url}
\begin{document}
\maketitle
\thispagestyle{fancy}
\begin{abstract}
This working paper presents a simple model for the macroeconomic
behavior of bitcoin based on the economic equation of exchange.
According to this model, the value of bitcoin is determined largely by
the willingness of bitcoin holders to save bitcoin and not by its
transactional use.  This model therefore predicts that increased use
of bitcoin will not cause its value to rise, but that the value of
bitcoin in terms of fiat currency will be almost solely determined by
the willingness of bitcoin holders to pull bitcoin out of circulation.
This model suggests that bitcoin will not fall victim to a liquidity
trap as suggested by some economists.
\end{abstract}

\begin{multicols}{2}

\section{Introduction} 

In this working paper, we present a model for the valuation of bitcoin
based on simple macroeconomic arguments.  We caution that our model is
intended primarily to stimulate discussion and that the truth of this
model or any economic model will be determined by empirical
observation of the bitcoin market.

\section{Macroeconomics of Bitcoin}

We begin our money with the equation of exchange where
\begin{equation}
M \cdot V = P \cdot Q
\end{equation}
where
\begin{eqnarray*}
& M & \mbox{is the nominal amount of money}\\
& V & \mbox{is the velocity of money}\\
& P & \mbox{is the price level}\\
& Q & \mbox{is the index of real expenditures}\\
\end{eqnarray*}

We now modify the equation in order to take into account the unique
aspects of bitcoin.  First we express all of our quantities in units
of fiat currency.  By expressing all of our quantities in terms of
fiat currency, we can set the $P$ to $1$.  Since we are expressing all
quantities in units of fiat currency, the value for $M$ is now the
value of bitcoin as measured in fiat currency units.

We next expand the quantity $M$ in terms of the number of bitcoin in
circulation $n_b$ and the price of a single bitcoin $p_b$.  The number
of bitcoin in circulation is externally determined and slowing and
predictable varying, whereas the price of a single bitcoin will
fluctuate.  Substituting $M=n_b p_b$ we and rearranging we get.

\begin{equation}
p_b = \frac{Q}{n_b V}
\end{equation}

We note that $n_b$ is an externally determined and slowing changing
variable.  The main determinant of the price of bitcoin is the
interaction between the level of bitcoin usage and the velocity of
bitcoin.

At this point we have made no assumptions concerning the dynamics of
bitcoin.  We shall now add some dynamical assumptions concerning the
relationship between the number of bitcoin expended $Q$ and the
velocity of an individual bitcoin $V$.

We decompose the velocity of bitcoin into a portion that is saved and
portion that is transacted.  We denote the likelihood that a
invididual bitcoin is saved by $l_s$ and the likelihood that a bitcoin
is transacted as $l_t$.  These two variables will sum to $1$.  These
portions have a corresponding set of velocities $v_s$ being the
velocity of a saved bitcoin and $v_t$ being the velocity of a
transacted bitcoin.

We now make several claims about the dynamics of bitcoin.  Because
$v_t \gg v_s$ we claim that we can set the velocity of saved bitcoins
to $0$.  We further claim that the velocity of transacted bitcoins can
be modeled as a linear function of $Q$.  We therefore have the
following expression for the total velocity of bitcoin.

\begin{equation}
V = l_t\alpha_t Q
\end{equation}.

Substituting back into the original equation we see that
\begin{equation}
p_b = \frac{1}{l_t \alpha_t}
\end{equation}.

We claim that $\alpha_t$ will remain roughly constant over time.  That
leaves only one term which impacts the price of bitcoin, which is the
$l_t$ the likelihood that a given bitcoin will be transacted rathr
than saved.

\section{Implications of our model}
Our model makes very strong claims concerning the price of bitcoins.
In particular, it states that the price of bitcoin is determined
almost solely by the likelihood that a given bitcoin will be saved. If
a user uses bitcoin for transaction purposes then this has no impact
on the value of bitcoin, while the value of bitcoin rises a given
bitcoin is more likely to be saved rather than transacted.

This model seems to agree with the changes in bitcoin pricing in 2013.
The two events that caused bitcoin prices to increase were the Cyprus
banking crisis in April 2013 and the rise of mainland Chinese bitcoin
exchanges in October 2013.  Both these events increased the likelihood
that a given bitcoin would be saved rather than transacted which
increased the price of bitcoin.  We note that the main drivers for
bitcoin prices have been news reports concerning the convertibility of
bitcoin, and that news reports concerning the increasing usage of
bitcoin has made very little impact on its price.

We also note that bitcoin has is prone to sudden changes in price due
to news events but that the price of bitcoin after a sudden shock has
returned to stable levels.  Our model explains this phenomenon by
asserting that although shocks can cause a sudden change in the price
of bitcoin, to the extent that it does not cause a change in how
likely a bitcoin is saved or transacted, it will not cause a change in
the long-run price level of bitcoin.

Our model also makes predictions concerning the future price of
bitcoin.  If our model is correct, then we should expect that
increased usage of bitcoin should manifest itself in larger volumes
rather than increased prices.  Prices in bitcoin would only be
effected by events which increase or decrease the likelihood that a
given bitcoin would be saved rather than transacted.

Some economists, notably Paul Krugman, are pessimistic about the
viability of bitcoin as a medium of exchange, because they believe
that bitcoin will fall into a liquidity trap caused by the fixed
supply of bitcoin \cite{krugman1}.  Krugman claims that as the price
of bitcoin rises, the amount of bitcoin in circulation will decrease
thereby increasing the price, ultimately creating a situation in
which bitcoin is completely hoarded.  This effect has been observed
in scrip economies such as the Capital Hill Babysitting Co-op 
\cite{krugman2}.

However, our model provides a mechanism by which bitcoin can avoid a
liquidity trap and explains why bitcoin has not experienced such a
trap.  The mechanism which we claim this will occur involves the
interaction between bitcoin and fiat currency.  As the price of
bitcoin rises, we claim that people will be more likely to spend their
bitcoin rather than fiat currency, which will decrease saved bitcoin
thereby cause bitcoin prices to fall.  Conversely a fall in the price
of bitcoin will increase the likelihood that people will save bitcoin
as they will be more likely to spend fiat, and this will cause bitcoin
prices to rise.  Our models claims that long-run increases or
decreases in the price of bitcoin will not influenced by the
transactional use of bitcoin but rather by external factors which
change the likelihood that a given bitcoin will be saved.

Krugman's belief that bitcoin is destined for a liquidity trap ignores
the fact that bitcoin exists within an fiat-based economy, and that
the ease of convertibility to and from fiat prevents a liquidity trap
that exists when convertibility is difficult.

\section{Conclusion}
We caution that the correctness of the model will be determined by
future empirical observations of the bitcoin market.  However, we hope
to illustrate that the behavior of bitcoin can be modelled using basic
economic principles, and we hope that this working paper will
stimulate research and discussion in this area.

\bibliography{macro}
\end{multicols}

\end{document}
