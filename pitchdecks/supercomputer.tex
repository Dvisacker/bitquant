\documentclass{beamer}
\title {Blockchain for High Performance Computing}
\author{Joseph Chen-Yu~Wang}
\institute{Bitquant Research Laboratories (Asia) Limited}
\date{\today}
\usetheme{Dresden}
\usecolortheme{beaver}
\begin{document}
\frame{\titlepage}
\begin{frame}
  \frametitle{About me}
Professional supercomputer babysitter.  Ph.D. in computational
astrophysics, worked as a quant in parallel computing 
infrastructure at JPMorgan.
\end{frame}
\begin{frame}
  \frametitle{Blockchain supercomputer}
  \begin{itemize}
  \item The world's largest computer is being used for bitcoin mining.  How do
    we using blockchain to create a general purpose supercomputer.  
  \item Fastest supercomputers consist of clusters of standard
    supercomputing components.
  \item Most large companies have large amounts of internal compute
    power which is not being used
  \end{itemize}
\end{frame}
\begin{frame}
  \frametitle{The Basic Idea}
You use a blockchain network to offer cash bounties on whoever is
the first person to solve a math equation.  The blockchain network
registers and verifies solutions.

The important idea here is that the ethereum network is extremely
slow, and will take a very long time to verify the calculation.  However,
this does not matter, since the role of the ethereum system is to
verify who gets the cash bounty.  It does not matter how long it takes
the blockchain to do the calculation as long as the cash is eventually
sent to the winner.
\end{frame}
\begin{frame}
  \frametitle{More details}
  The asker registers a math problem onto the ethereum blockchain and
  a payment which is escrowed.  The blockchain begins to do the
  calculation very, very slowly, while recording the proposed
  solutions that are received in order of priority.
  
  When the ethereum calculation is finished, it will forward the payment
  to the first N systems with the correct answer.
\end{frame}
\begin{frame}
  \frametitle{Solution deposit}
  A person with a proposed solution must agree to pay a deposit.  The
  deposit will be used as gas to verify the solution.  If the solution
  is correct, the deposit will be returned to the solution provider.
  If the solution is incorrect, the deposit will be given to the the
  poser.  The purpose of this system is to avoid spamming of incorrect
  answers.

  This system also has the added benefit that the poser of the problem
  can look at the proposed solutions, and use the correct solution
  before the ethereum network has verified the solution.
\end{frame}  
\begin{frame}
  \frametitle{Asker interface}
  Asker would upload to the blockchain following pieces of information
  \begin{itemize}
  \item The cash bounty for solving the problem which will be
    escrowed and how the bounty is to be distributed among the first
    N correct answers.
  \item A minimum deposit for proposed solutions
  \item The math problem to be solved
  \item A timeout limit at which point the bounty will be canceled
    and the cash will be returned to the originator.  The timeout will
    only return the coin if no correct solution has been offered
    within the timeout.
  \end{itemize}
\end{frame}
\begin{frame}
  \frametitle{The Math Specification}
  \begin{itemize}
    \item Set of an equation.  If you make it equal zero then you get
      the bounty
    \item Use metadata and conventions to allow solvers to determine
      what programs to solve
    \item Equations can be disguised to remove sensitive information
      (i.e. add random numbers to the problem to be solved)
  \end{itemize}
\end{frame}
\begin{frame}
  \frametitle{Solver interface}
The solver would solve the problem in any possible way.
  \begin{itemize}
  \item A digital currency address to send the bounty
  \item Proposed solution - The proposed solution can be sent
    immediately to the user.  The verification on the blockchain is
    needed only for the bounty
  \item A deposit which will be returned to the solver if the answer
    is correct or given to the asker if the answer is not.
  \end{itemize}
\end{frame}
\begin{frame}
  \frametitle{Bootstraping the system}
  \begin{itemize}
    \item Have a simple framework that allows registration of problems and
      funds transfers
    \item Create reference API
    \item Create simple network to allow for compute calculations
    \item Companies can create a supercomputer with no hardware outlay
    \item One developer - Three Months of work
  \end{itemize}
\end{frame}
\begin{frame}
  \frametitle{Extending the system}
  \begin{itemize}
  \item Create floating point system in ethereum
  \item Interface with standard math libraries (BLAS and LAPACK)
  \item Add supercomputing centers to the system
  \item Add better meta-data for certain types of problems
  \end{itemize}
\end{frame}
\begin{frame}
  \frametitle{Comparison with other systems}
  \begin{itemize}
    \item Golem, Zennet, 21.co and iex.ec are asking for millions of
      dollars.  This mechanism can be implemented in a computer
      cluster by one developer.
    \item Every other system will take a year and massive spend before
      you know if it works.  This system will have something up and
      running in a matter of one to two months.
    \item This system will allow you to distribute compute tasks
      across a small network. I can use it to off load compute from my
      laptop to my big server.
    \item This system will allow for ``stealth marketing'' into the
      enterprise.  If you want a problem solved, just upload the
      problem, and get an answer.
  \end{itemize}
\end{frame}
\begin{frame}
    \frametitle{Comparison with other systems}
    \begin{itemize}
    \item I do not understand the the business model for the other
      projects.  In this architecture, the transaction layer makes no
      money.  My business model is that if I have access to the
      world's biggest supercomputer, I'll find some way of making
      money as a user
    \item All of the other blockchain supercomputer proposals assume
      that people are going to be using desktop systems.  This system
      distributes math problems and doesn't care what mechanism is
      used to solve them.
    \item Assumption that once a market exists that there will be
      enough money involved that people will develop special purpose
      hardware (i.e. GPU's, hardware systems, quantum computers) to
      collect the bounty.  
    \end{itemize}
\end{frame}
\begin{frame}
  \frametitle{What I need to make this happen}
  \begin{itemize}
  \item Estimated cost: USD 10,000/month for three to six months of work
  \item Will allow a company to set up a compute farm
  \item Alternatively, if someone can help me do an ICO that can
    raise USD 6 million (which is what some of the other projects
    have raised) we can it that way.
  \item Alternatively, I really do not think that the approaches that
    the other systems are taking will work.  If someone can convince
    them to give up what they are doing, and to implement this, that
    will work for me.
  \end{itemize}
\end{frame}
\end{document}
