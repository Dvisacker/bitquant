\documentclass{beamer}
\title {Blockchain for High Performance Computing}
\author{Joseph Chen-Yu~Wang}
\institute{Bitquant Research Laboratories (Asia) Limited}
\date{\today}
\usetheme{Dresden}
\usecolortheme{beaver}
\begin{document}
\frame{\titlepage}
\begin{frame}
  \frametitle{About me}
Professional supercomputer babysitter.  Ph.D. in computational
astrophysics, worked as a quant in parallel computing 
infrastructure at JPMorgan.
\end{frame}
\begin{frame}
  \frametitle{Blockchain supercomputer}
  \begin{itemize}
  \item The world's largest computer is being used for bitcoin mining.  How do
    we using blockchain to create a general purpose supercomputer.  
  \item Fastest supercomputers consist of clusters of standard
    supercomputing components.
  \item Most large companies have large amounts of internal compute
    power which is not being used
  \end{itemize}
\end{frame}
\begin{frame}
  \frametitle{The Basic Idea}
  You use a blockchain network to offer cash bounties on whoever is
  the first person to solve a math equation.  The blockchain network
  registers and verifies solutions, but does not to the calculation.
\end{frame}
\begin{frame}
  \frametitle{Why this works}
Most math problems of interest are set up that the solution is easy to
verify but hard to calculate.  The blockchain network would be used
only to register bounties to solve math problems and distribute cash
bounties.
\end{frame}
\begin{frame}
  \frametitle{User interface}
  User would upload to the blockchain three pieces of information
  \begin{itemize}
  \item The cash bounty for solving the problem which will be escrowed.
  \item The math problem to be solved
  \item A timeout limit at which point the bounty will be canceled
    and the cash will be returned to the originator.
  \end{itemize}
\end{frame}
\begin{frame}
  \frametitle{The Math Specification}
  \begin{itemize}
    \item Set of an equation.  If you make it equal zero then you get
      the bounty
    \item Use metadata and conventions to allow solvers to determine
      what programs to solve
    \item Equations can be disguised to remove sensitive information
      (i.e. add random numbers to the problem to be solved)
  \end{itemize}
\end{frame}
\begin{frame}
  \frametitle{Solver interface}
The solver would solve the problem in any possible way.
  \begin{itemize}
  \item A digital currency interface to send the bounty
  \item Proposed solution - The proposed solution can be sent
    immediately to the user.  The verification on the blockchain is
    needed only for the bounty
  \end{itemize}
\end{frame}
\begin{frame}
  \frametitle{Bootstraping the system}
  \begin{itemize}
    \item Have a simple framework that allows registration of problems and
      funds transfers
    \item Create reference API
    \item Create simple network to allow for compute calculations
    \item Companies can create a supercomputer with no hardware outlay
    \item One developer - Two Months of work
  \end{itemize}
\end{frame}
\begin{frame}
  \frametitle{Extending the system}
  \begin{itemize}
  \item Create floating point system in ethereum
  \item Interface with standard math libraries (BLAS and LAPACK)
  \item Add supercomputing centers to the system
  \item Add better meta-data for certain types of problems
  \end{itemize}
\end{frame}
\end{document}
