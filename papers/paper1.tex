\title{Hong Kong Reguation of Bitcoin}
\author{Joseph C Wang\\
\small Bitquant Research Laboratories
}
\begin{document}
\begin{abstract}
In recent months, bitcoin has received a lot of attention in the
media, and there has been much discussion online concerning the
regulatory aspects of bitcoin.  In this article, we survey the role of
bitcoin within the financial regulatory structure of Hong Kong playing
special attention to Hong Kong's unique position as an international
financial center within the People's Republic of China.  We examine
the relevant laws and regulation in Hong Kong concerning bitcoin and
the implications of bitcoin's designation as a virtual commodity.  We
then example the purpose of financial regulation and argue that the
current regulatory structure is well suited to those goals.  We then
examine the possible future of the regulation of bitcoin and list a
number of areas for attention by regulators and provide
recommendations for bitcoins possible future.
\end{abstract}

\section{Introduction}

Bitcoin is one of several cryptocurrencies which received much
addition in the media in late 2013.  Among the reasons for this was
the increasing use of bitcoin in Mainland China, the dramatic increase
in price, and the volatility following regulatory announcements by the
People's Bank of China.

The future of bitcoin is uncertain with opinions on the net ranging
from positive to highly negative.  Many applications for bitcoin have
been suggested involving money transfer.

\section{Hong Kong's Regulatory Environment}
Hong Kong has a unique position as a financial center which causes
financial regulation within Hong Kong to be different from those of
other jurisdiction.  Under the British rule, Hong Kong became a
financial trading center, and this position has been entrenched in
Hong Kong's mini-constitution, the Basic Law of Hong Kong which has
governed H

Specifically Article **** of the Basic Law states that

In addition to the formal goals of Hong Kong in law, there are
informal goals that define the system of Hong Kong.  Since the 1960's
and following through after the handover the regulatory policy of Hong
Kong can has been described as ``positive non-interventionism'' in
which the government relies on the aggregate judgment of business, and
avoids intervening in with the market unless there is a clear social
need to do so.

The philosophy of ``positive non-interventionism'' can be see in the
financial regulatory structure of Hong Kong.  Rather than create a
general regulatory framework that encompasses all financial
activities, Hong Kong financial regulation starts with the presumption
that financial activities should be unregulated and subject only to
common law which defines the economic roles of financial actors.  When
there is a compelling need to do so, the government will issue
regulations, but these regulations have been very narrowly tailored to
address the social need and are designed to leave as much of the
market as unregulated as possible.

\section{Interaction with Mainland China}
While maintain it's separate financial system, Hong Kong is a part of
the People's Republic of China, and part of one country, Hong Kong has
played in integral role in the development of the Mainland Chinese
economy.

The Mainland Chinese authorities have set as a target 2020 for full
convertibility of the Renminbi.

\section{Regulation of Bitcoin in Hong Kong}

The regulation of Bitcoin both in Mainland China and in Hong Kong has
taken a unique and somewhat unusual route.  Rather than creating a new
set of regulations regarding bitcoin, both Mainland China and Hong
Kong have sought to define the status of bitcoin allowing it to be
regulated under current regulations.

The status of bitcoin in Mainland China has been formally defined in
the ``Notice....''  In this notice, the People's Bank of China has
stated.  

The definition of bitcoin as a virtual commodity has implications for
Hong Kong.  By defining bitcoin as a non-currency, there are no
restrictions on trading of cross-border trading of bitcoin between
Hong Kong and the PRC.  In addition, the restrictions on bitcoin
effectively sets a floor on regulation in Hong Kong.  For Hong Kong to
adopt restrictions on bitcoin that do not exist in Mainland China
would drive bitcoin business from Hong Kong to the Mainland, and this
would contravene Hong Kong's role as an international financial center
and as an a free zone for the PRC.

The attitude of Hong Kong regulators has matched those of the PRC.  In
a press release, the HKMA has stated.

*****

And in response to a Legco question, it has been stated

******

The definition of bitcoin as a virtual commodity in both Mainland
China and Hong Kong greatly clarifies its status within the regulatory
system of HK and Mainland China and allows for small companies to
build businesses around bitcoin.  We shall now examine individual laws
within Hong Kong and see the implications of bitcoin's definition as a
virtual currency and the differences of this definition that want
would be the case if bitcoin is classified as currency or as a security.

\subsection{Commodity Markets Prohibition Ordinance}
As a commodity, bitcoin is potentially subject to the commodity
markets prohibition ordinance, which prohbits the creating of
commodity markets for certain commodities listed in Schedule 1 of the
ordinance.  Bitcoin is not listed in Schedule 1, however the Chief
Executive of Hong Kong is empowered to add commodities onto the list.
The effect of the Commodity Markets Prohibition Ordinance is to allow
the HKSAR to ban the trading of bitcoin on an exchange by order of the
chief executive.

\subsection{Deceptive Claims Ordinance}

The application of the deceptive claims ordinance to commodities can
be seen specifically in relation to the special provisions regarding
gold coin.

\subsection{Theft Ordinance}



\subsection{Exchange Fund Ordinance}

\subsection{Anti-Money Laundering and Counter-Terrorist Financing
(Financial Institutions) Ordinance Cap 615}

This ordinance which was passed in 2012 has two parts.  The first
section imposes record keeping requirements on all financial
institutions.  The second section creates a system for licensure for
money services, which are defined under Cap 615 s 24 and includes the
following definitions

currency - includes a cheque and a traveller's cheque

money means money in whatever form or currency

money changing service means a service for the exchaging of currencies
that is operaated in Hong Kong as a business 

money service means: a money changing service or a remittance service

The license system for a money services business is adminstered by the
Department of Customs and Excise, and the criterion for the issuance
of a license are those that related to identifying the ultimate
beneficial owners of the money service and whether the owners have
been convicted of money laundering, drug trafficking, terrorism or
other serious crime.  The criterion for license issuance are criminal
and character based and do not include the financial stability or
creditworthness of the business.

Based on the definitions in this ordinance, and the definition of
bitcoin as a commodity, no license appears to be necessary for a
business whose sole function is to convert bitcoin to and from Hong
Kong dollars, and the ordinance appears not to include gold dealers
and jewellers operating in Hong Kong.

However, a license may be necessary if the business intends to use
bitcoin as a mechanism to convert currencies within Hong Kong or to
transmit money outside of Hong Kong.

Once a company becomes a licensed money services operator, they are
then considered a financial institution, and must abide by the record
keeping and other provisions of the first section of the ordinance.

Although a money service license is not essential for operating a
bitcoin exchange, several of the exchanges in Hong Kong have applyed
for and received money services license, and have undertaking AML-KYC
examinations. 

\subsection{Securities and Futures Ordinance}
The most complicated interactions between bitcoin and the Hong Kong
regulatory system come with the interaction with the Securities and
Futures Ordinance.  While it is clear that as a virtual commodity the
SFO does not impact the trading of bitcoin directly, the SFO may set
some limitations on the trading of products related to bitcoin.

\subsection{Definitions under SFO}

The SFO creates a regulatory structure by listing a number of
regulated activities concerning ``securities'' and ``futures.''  Hence
an analysis of the status of bitcoin under the SFO must begin with the
definitions of securities, futures, futures market, and automated
trading system

* definition of securities
* definition of futures
* definition of futures market
* definition of automated trading system

\subsection{Status of bitcoin under the SFO}

Given the definitions in the SFO, it is clear that currently bitcoin
and other cryptocurrencies as virtual commodities are neither
``securities'' or ``futures'' and therefore are not subject to the
securities futures ordinance.  As bitcoin is not subject to the SFO,
businesses such as exchanges, brokerage, and operation of automated
trading systems of bitcoin are not regulated activities subject to the
SFO.  However, this situation can change by regulatory action.  Under
Section 392, the financial secretary has the authority define bitcoin
as a security thereby subjecting it to SFO.  While the financial
secretary has this authority, this authority has not be exercised, and
as we shall argue below, it would be unwise for the Financial
Secretary to do so.

\subsection{Status of bitcoin derivatives}
Although bitcoin itself is not subject to SFO, there are two possible
ways in which a bitcoin derivative could be subject to the SFO.
Either the derivative could be classified as a ``structured product''
and therefore a security, or it could be classified as a ``future.''

The SFO defines a ``structured product'' as ....

The regulation of structured products is a relatively new addition to
the SFO which was enacted in the aftermath of the Lehman mini-bonds
debacle.  The precise definition of structured product reflects these
origins as it covers only those products in which the price level is
set in the contract thereby exposing the buyer to counterparty risk.
Under the terms of the SFO as currently written, a future or option
would not be considered a structured product as the value of said
product not stated in the contract, but is the result of the value
given by the market.

The definition of future within the SFO is.....

Critical in the definition of ``futures market'' is the inclusion of
place or facility.  It can be questionable whether a market without a
physical location can be considered to be a ``futures market'' under
the terms of ordinance.  (**** find current licenses examples ****)

\subsection{Bitcoin Exchanges}

The SFO also prohibits a Hong Kong company from using the following terms....

\subsection{Bitcoin Automated Trading Systems}

\subsection{False Trading}

\section{Goals of Regulation}

\subsection{International financial center}

The nature of the regulation regarding bitcoin is also consist with
the need to preserve Hong Kong's system of free enterprise, and a
business environment which is conducive to innovation.

The decision by the HK and PRC regulatories to define bitcoin within
the framework of existing law is positive as this existing laws are
known and businesses can figure out what the implications are.  In
addition, the definition allows ordinary people to think about bitcoin
in a way that does not require a lawyer.  Essentially, bitcoin can be
thought of as a gold coin, and legal transactions that are valid with
gold coins are generally valid with bitcoin and vice versa.

\subsection{Macroeconomic stability}

\subsection{Investor protection}
There are currently no investor protections in Hong Kong concerning
bitcoin beyond those provided by general anti-fraud and anti-deception
statutes.  In order to examine whether this is an adequate regulatory
framework we must first examine why it is necessary to adopt special
rules regarding securties.

deterence is sufficient

fraud laws are focused on particular acts and not generalized harm

conflict of interest results in imprudence





\subsection{Anti money and anti-terrorism}

\section{Future Directions in Bitcoin Regulation}

\subsection{Anti-Money and Anti-Terrorism Laundering}

Because the rules are applicable to all financial institutions from
large to small, it appears unlikely that bitcoin will outgrow these
regulations regardless of the outcome.  If any changes to the AML
infrastructure is necessary, it will be an item that influences all
financial institution.  Specific regulations on bitcoin will only
prove necessary if bitcoin has some as yet unknown characteristic that
encourages money laundering and criminal activity is a manner that
regular currency does not.

\subsection{Macroeconomic stability}

The main concern in this area is that bitcoin will be part of a
shadow banking system in much the same way financial derivatives
created such as system before the 2008 financial crisis.

The effort to prevent this has consisted of separating the bitcoin
economy from the banking system.  

In determining whether further action is necessary it will be
sufficient to look at the total volume of bitcoin.  If a substantial
amount of economic activity becomes conducted in bitcoin, then further
legislation may be necessary.  In addition, further thinking as to the
regulatory structure of bitcoin may be necessary if financial
institutions desire entering the bitcoin business.

Additional gudiance from the HKMA may also be necessary as to the
types and quantity of business loans made to support bitcoin.


\end{document}
